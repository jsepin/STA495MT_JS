\documentclass[11pt, a4paper]{article}\usepackage[]{graphicx}\usepackage[]{xcolor}
% maxwidth is the original width if it is less than linewidth
% otherwise use linewidth (to make sure the graphics do not exceed the margin)
\makeatletter
\def\maxwidth{ %
  \ifdim\Gin@nat@width>\linewidth
    \linewidth
  \else
    \Gin@nat@width
  \fi
}
\makeatother

\definecolor{fgcolor}{rgb}{0.345, 0.345, 0.345}
\newcommand{\hlnum}[1]{\textcolor[rgb]{0.686,0.059,0.569}{#1}}%
\newcommand{\hlstr}[1]{\textcolor[rgb]{0.192,0.494,0.8}{#1}}%
\newcommand{\hlcom}[1]{\textcolor[rgb]{0.678,0.584,0.686}{\textit{#1}}}%
\newcommand{\hlopt}[1]{\textcolor[rgb]{0,0,0}{#1}}%
\newcommand{\hlstd}[1]{\textcolor[rgb]{0.345,0.345,0.345}{#1}}%
\newcommand{\hlkwa}[1]{\textcolor[rgb]{0.161,0.373,0.58}{\textbf{#1}}}%
\newcommand{\hlkwb}[1]{\textcolor[rgb]{0.69,0.353,0.396}{#1}}%
\newcommand{\hlkwc}[1]{\textcolor[rgb]{0.333,0.667,0.333}{#1}}%
\newcommand{\hlkwd}[1]{\textcolor[rgb]{0.737,0.353,0.396}{\textbf{#1}}}%
\let\hlipl\hlkwb

\usepackage{framed}
\makeatletter
\newenvironment{kframe}{%
 \def\at@end@of@kframe{}%
 \ifinner\ifhmode%
  \def\at@end@of@kframe{\end{minipage}}%
  \begin{minipage}{\columnwidth}%
 \fi\fi%
 \def\FrameCommand##1{\hskip\@totalleftmargin \hskip-\fboxsep
 \colorbox{shadecolor}{##1}\hskip-\fboxsep
     % There is no \\@totalrightmargin, so:
     \hskip-\linewidth \hskip-\@totalleftmargin \hskip\columnwidth}%
 \MakeFramed {\advance\hsize-\width
   \@totalleftmargin\z@ \linewidth\hsize
   \@setminipage}}%
 {\par\unskip\endMakeFramed%
 \at@end@of@kframe}
\makeatother

\definecolor{shadecolor}{rgb}{.97, .97, .97}
\definecolor{messagecolor}{rgb}{0, 0, 0}
\definecolor{warningcolor}{rgb}{1, 0, 1}
\definecolor{errorcolor}{rgb}{1, 0, 0}
\newenvironment{knitrout}{}{} % an empty environment to be redefined in TeX

\usepackage{alltt}
\usepackage{geometry}
\geometry{
  left=2cm,
  right=2cm,
  top=2cm,
  bottom=2cm,
}
\usepackage{amsmath}
\usepackage{booktabs}
\usepackage{multicol}
\usepackage{enumitem}
\usepackage{graphicx}
\usepackage[latin1]{inputenc}
\usepackage{tikz}
\usetikzlibrary{shapes,arrows}
\usetikzlibrary{arrows.meta}
\usepackage{varwidth}
\usetikzlibrary{positioning}
\IfFileExists{upquote.sty}{\usepackage{upquote}}{}
\begin{document}
\pagestyle{empty}
\footnotesize


% Load parameters



\pagebreak



%%% Simulation visualization
% Define block styles
\tikzstyle{decision} = [diamond, draw, fill=blue!20, 
    text width=4.5em, text badly centered, node distance=3cm, inner sep=0pt]
\tikzstyle{block} = [rectangle, draw, fill=blue!15, 
    text width=5em, text centered, rounded corners, minimum height=4em]
\tikzstyle{dg} = [rectangle, draw, fill=green!15, rounded corners, minimum height=4em]  
\tikzstyle{line} = [draw, -latex']
\tikzstyle{cloud} = [draw, ellipse,fill=red!20, node distance=3cm,
    minimum height=2em]
\tikzstyle{rect} = [draw, rectangle,fill=orange!80, node distance=3cm,minimum height=2em]


%%%%%%%%%%%%%%%%%%%%%%%%%%%%%%%%%%%%%%%%%%%%%%%%%%%%%%%%%%%%%%%%%%%%%%%%%%%%%%%%%%%%%%%%%%%%%%%%%%%%%%%%%%%%%%%%%%%%%%%%%%%
\tikzstyle{abstract}=[rectangle, draw=black, rounded corners, fill=blue, text centered, anchor=north, text=white, text width=3cm]
\tikzstyle{comment}=[rectangle, draw=black, rounded corners, text centered, anchor=north, text=black,text width=9cm]
% \tikzstyle{myarrow}=[->, >=open triangle 90, thick]
% \tikzstyle{line}=[-, thick]
%%%%%%%%%%%%%%%%%%%%%%%%%%%%%%%%%%%%%%%%%%%%%%%%%%%%%%%%%%%%%%%%%%%%%%%%%%%%%%%%%%%%%%%%%%%%%%%%%%%%%%%%%%%%%%%%%%%%%%%%%%%
%\begin{figure}[h]%H is strict!
%\begin{center} 
\tikzstyle{every node}=[font=\footnotesize]
\tikzset{every picture/.style={line width=0.75pt}} %set default line width to 0.75pt        
\begin{tikzpicture}[x=0.75pt,y=0.75pt,yscale=-1,xscale=1]
%uncomment if require: \path (0,748); %set diagram left start at 0, and has height of 748


% True linear model rectangle
\draw  [fill=yellow!40  ,fill opacity=1, line width=2] (0,30) -- (450,30) -- (450,90) -- (0,90) -- cycle ;
\draw (0,30) node [anchor=north west][inner sep=0.75pt]   [align=left] {\textbf{True linear model}};
%\path [line] (150,90) -- (150,580);


% Text Node (Jerome)
\node [cloud, anchor = center] at (60,60) (X_sym) {$\ \ \ \mathbf{X} \ \ \ $};
\draw (100,60) node [anchor=center][inner sep=0.75pt]   [align=left] {$ \mathbf{\cdot}$};
\node [rect, anchor = center] at (130,60) (beta) {$\ \ \ \mathbf{\beta} \ \ \ $};
\draw (190,60) node [anchor=center][inner sep=0.75pt]   [align=left] {+};
\node [cloud, anchor = center,fill=gray!10] at (250,60) (coll) {$\ \ \ \boldsymbol{\varepsilon _{y}} \ \ \ $};
\draw (290,60) node [anchor=center][inner sep=0.75pt]   [align=left] {$ \mathbf{\cdot}$};
\node [rect, anchor = center,fill=gray!10] at (320,60) (coll) {$\ \ \ s_{y} \ \ \ $};
\draw (360,60) node [anchor=center][inner sep=0.75pt]   [align=left] {=};
\node [cloud, anchor = center,fill = yellow!20] at (405,60) (coll) {$\ \ \ \boldsymbol{y} \ \ \ $};

% Conclusions
\node [rect, anchor = west,fill = yellow!20] at (470,60) (conclusion) {\begin{varwidth}{\linewidth}
\textbf{Conclusions}
\end{varwidth}};

% Design Matrix (Inducing Collinearity)
\node [dg, anchor = south west,fill=red!20] at (0,20) (X) {\begin{varwidth}{\linewidth}
\textbf{Design Matrix (Collinearity):}
\begin{itemize}[leftmargin=10pt]
  \item \texttt{rho<-tanh(seq(\\
  from=-1.8,to=0,\\
  length.out=20))}
  \item \texttt{n\_obs<-c(5, 31, 400)}
  \item Distribution of $\boldsymbol{X}$ (normal)\\ 
        \texttt{
        X<-mvtnorm::rmvnorm(\dots)
        }
  \item \texttt{mean\_x1<-0.6; mean\_x2<-3.8}
  \item \texttt{sd\_x1<-0.1; sd\_x2<-2.1}
  \item Number of simulations $B$\texttt{<-1211}
  \end{itemize}\end{varwidth}
};

% Aim
\node [rect, anchor = north west,fill = red!70, line width= 0.8 mm] at (210,-160) (aim) {\begin{varwidth}{\linewidth}
\textbf{Aim:}\\ How does collinearity influence the output\\($\boldsymbol{\hat{t},\hat\beta, \text{se}(\hat\beta)}$,power) of \texttt{lm}, \texttt{tram::Lm} ?
\end{varwidth}};

% Regression
\node [dg, anchor = south west,fill=orange!80] at (210,20) (regression) {\begin{varwidth}{\linewidth}
\textbf{Regression parameters:}
\begin{itemize}[leftmargin=10pt]
  \item \texttt{beta\_0<-c(51.4)}
  \item \texttt{beta\_1<-c(-46.1, 0)}
  \item \texttt{beta\_2<-c(-0.9, 0)}
  \end{itemize}\end{varwidth}
};

% Noise
\node [dg, anchor = south west,fill=gray!10] at (375,20) (regression) {\begin{varwidth}{\linewidth}
\textbf{Noise parameters:}
\begin{itemize}[leftmargin=10pt]
  \item \texttt{set.seed(\dots)}
  \item \texttt{eps\_y<-rnorm(0,1,n=n\_obs)}
  \item \texttt{s\_y<-c(2, 5, 8.2)}
  \end{itemize}\end{varwidth}
};

%%%%%%%%%%%%%%%%%% underneath %%%%%%%%%%%%%%%%%%%


% Data-Generating-Process
\node (dgp) [comment, rectangle split, rectangle split parts=3, rectangle split part fill={white,red!20,yellow!20}, text justified, anchor = north west] at (0,110)
    {
        \textbf{Data-Generating-Process:}\\
        \texttt{
        for(k in 1:B)\{\\
        \hspace*{1mm} for(i in 1:nrow(experimental\_factors))\{\\
        \hspace*{1mm} \hspace*{1mm} X<-mvtnorm::rmvnorm(n = n\_obs[i],\\
        \hspace*{1mm} \hspace*{1mm} \hspace*{1mm} mean = c(mean\_x1,mean\_x2),\\
        \hspace*{1mm} \hspace*{1mm} \hspace*{1mm} sigma = matrix(c(\\
        \hspace*{1mm} \hspace*{1mm} \hspace*{1mm} sd\_x1\textasciicircum 2, rho[i]*sd\_x1*sd\_x2,\\
        \hspace*{1mm} \hspace*{1mm} \hspace*{1mm} rho[i]*sd\_x1*sd\_x2, sd\_x2\textasciicircum 2\\
        \hspace*{1mm} \hspace*{1mm} \hspace*{1mm} ), ncol = 2))\\
        \hspace*{1mm} \hspace*{1mm} X<-cbind(1,X)
        }
        \nodepart{second}
        \texttt{\hspace*{1mm} \hspace*{1mm}cond\_nu<-max(\\
        \hspace*{1mm} \hspace*{1mm}\hspace*{1mm} Collinearity::Var\_decom\_mat(X)[,"cond\_ind"])}\\
        \texttt{\hspace*{1mm} \hspace*{1mm}E<-Collinearity::equilibrate\_matrix(X)}\\
        \texttt{\hspace*{1mm} \hspace*{1mm}trouble<-diag(solve(t(E)\%*\%E))}
        
        \nodepart{third}
        \texttt{
        \hspace*{1mm} \hspace*{-1mm} eps\_y<-rnorm(0,1,n=n\_obs[i])\\
        \hspace*{1mm} \hspace*{1mm} y<-X\%*\%c(beta\_0[i],beta\_1[i],beta\_2[i])+eps\_y*s\_y[i]\\
        \hspace*{1mm} \hspace*{1mm} df\_list[[length(df\_list)+1]]<-data.frame(y,X[,-1])\\
        \hspace*{1mm} \}\\
        \}
        }
        
  };


% Experimental factors
\node [rect, fill=white!10, below = 20 of dgp.south west, anchor = north west] (expfact) {\begin{varwidth}{\linewidth}
\textbf{Experimental factors (full factorial):}\\
\texttt{experimental\_factors<-expand.grid(\dots)}\\
\scalebox{0.9}{
% latex table generated in R 4.2.2 by xtable 1.8-4 package
% Tue Jan 17 20:52:38 2023
\begin{tabular}{lllllll}
  \toprule
id & n\_obs & rho & beta\_0 & beta\_1 & beta\_2 & s\_y \\ 
  \midrule
1 & 5 & -0.9 & 51.4 & -46.1 & -0.9 & 8.2 \\ 
  1212 & 31 & -0.9 & 51.4 & -46.1 & -0.9 & 8.2 \\ 
  ... & ... & ... & ... & ... & ... & ... \\ 
  301540 & 5 & -0.9 & 51.4 & -46.1 & -0.9 & 2 \\ 
  ... & ... & ... & ... & ... & ... & ... \\ 
  870710 & 400 & 0 & 51.4 & 0 & 0 & 5 \\ 
  ... & ... & ... & ... & ... & ... & ... \\ 
  871920 & 400 & 0 & 51.4 & 0 & 0 & 5 \\ 
   \bottomrule
\end{tabular}

}
\end{varwidth}
};


% Estimands
\node [rect, fill=white!10, right = 50 of expfact.north east, anchor = north west] (estimands) {\begin{varwidth}{\linewidth}
\textbf{Estimands:}\\
\textcolor{white}{\texttt{exp}}\\
\scalebox{0.9}{
% latex table generated in R 4.2.2 by xtable 1.8-4 package
% Tue Jan 17 20:52:38 2023
\begin{tabular}{lllll}
  \toprule
hbeta\_0 & se\_0 & t\_0 & ... & cond\_nu \\ 
  \midrule
-13 & 366.3 & 0 & ... & 122.1 \\ 
  9.1 & 64.6 & 0.1 & ... & 69.1 \\ 
  ... & ... & ... & ... & ... \\ 
  -51.1 & 21.1 & -2.4 & ... & 26.5 \\ 
  ... & ... & ... & ... & ... \\ 
  0 & 2.7 & 0 & ... & 15.5 \\ 
  ... & ... & ... & ... & ... \\ 
  -2.7 & 2.2 & -1.2 & ... & 13.8 \\ 
   \bottomrule
\end{tabular}

}
\end{varwidth}
};


% Estimand-Generating-Process:
\node [dg, below = 30 of estimands.south east, anchor = north east] (egp) {\begin{varwidth}{\linewidth}
\textbf{Estimand-Generating-Process:}\\
\texttt{
\# paired design\\
m<-lm(data,y$\sim$x1+x2)\\
m<-tram::Lm(data,y$\sim$x1+x2)
    }
\end{varwidth}
};

% Simulated Data sets
\node [rect,fill = white, below = 20 of expfact.south, anchor = north] (sim_data3) {\color{white}\begin{varwidth}{\linewidth}Simulated Data\\
(n = 871920)\\ \texttt{df\_list[[1]]}\end{varwidth}};
\node [rect,fill = white, below left = 10 and 10 of sim_data3.north , anchor = north] (sim_data2) {\color{white}\begin{varwidth}{\linewidth}Simulated Data\\
(n = 871920)\\ \texttt{df\_list[[1]]}\end{varwidth}};
\node [rect,fill = white, below left = 10 and 10 of sim_data2.north , anchor = north] (sim_data1) {\begin{varwidth}{\linewidth}Simulated Data\\
(n = 871920)\\ \texttt{df\_list[[1]]}\end{varwidth}};

% Performance Evaluation
\node [rect,fill = white!20, above = 20 of estimands.north east, anchor = south east] (PE) {\begin{varwidth}{\linewidth}
\textbf{Performance Evaluation:}\\
Metrics:
\begin{itemize}[leftmargin=10pt]
\item Wald statistics $\boldsymbol{\hat{t}}$
\item Bias $\boldsymbol{\hat{\beta}-\beta}$
\item Relative Bias $\left(\boldsymbol{\hat{\beta}-\beta}\right)/\boldsymbol{\beta}$
\item Empirical standard error $\text{se}\left(\boldsymbol{\hat{\beta}}\right)$
\item Proportion of $p$-values $\leq\alpha$ (Power)
\end{itemize}
Figures:
\begin{itemize}[leftmargin=10pt]
\item Trace plots with quantiles
\item Spatial plots
\end{itemize}
\end{varwidth}
} ;


%%%%%%%%%%%%%%%%%%%%%%%%%%%%%%%%%%%%%%%%%%%%%%%%%%%%%%%%%%%%%%%%%%%%%%%%%%%%%%%
% Edges
%%%%%%%%%%%%%%%%%%%%%%%%%%%%%%%%%%%%%%%%%%%%%%%%%%%%%%%%%%%%%%%%%%%%%%%%%%%%%%%
\path [line] (X_sym |- 0,90) -- (dgp.north-|X_sym.south);
\path [line] (beta.south) -- +(0,+25) -| (PE.north-|egp);
\path [line] (dgp.south) -- (dgp|-expfact.north);
\path [line] (sim_data1|-expfact.south) -- (sim_data1.north);
\path [line] (sim_data1.east) -- (sim_data1-|egp.west);
\path [line] (egp.north-|egp) -- (estimands.south-|egp);
\path [line] (estimands.north-|egp) -- (PE.south-|egp);
\path [line] (PE.north-|conclusion.south) -| (conclusion.south);
\path [line] (conclusion.east) -- +(+10,0) |- (aim.east);


%%%%%%%%%%%%%%%%%%%%%%%%%%%%%%%%%%%%%%%%%%%%%%%%%%%%%%%%%%%%%%%%%%%%%%%%%%%%%%%
\end{tikzpicture}
%\end{center}
% \caption{Simulation workflow regarding the parameter estimation process.}
% \label{fig:flow_para}
% \end{figure}
%%%%%%%%%%%%%%%%%%%%%%%%%%%%%%%%%%%%%%%%%%%%%%%%%%%%%%%%%%%%%%%%%%%%%%%%%%%%%%%

\end{document}
